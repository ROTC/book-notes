\chapter{Additive Models, Trees, and Related Methods}
    \section{Generalized Additive Models}
    \section{Tree-Based Methods}
        \subsection{Background}
        \subsection{Regression Trees}
        \subsection{Classification Trees}
            不同的节点非纯度度量${Q_m}(T)$,包括交叉熵或者deviance(散离)。

            交叉熵和基尼指数对结点概率的改变更加敏感,相对于错误率来说。
        \subsection{Other issues}
            {\color{blue} \bf{Categorical Predictors}}

            {\color{blue} \bf{The Loss Matrix}}

            观测的误分类后果对于某些类要比其他类严重。为了把损失引入到建模过程中,可以把Gini指数修改成$\sum\nolimits_{k \ne k'} {{L_{kk'}}} {\hat p_{mk}}{\hat p_{mk'}}$ 。该方法对多分类比较有效,对于二分类,系数不起作用,更好的办法是给$k$类中的样本加权$L_{kk'}$。对于多分类来说,仅当$L_{kk'}$与$k'$无关时才能使用。观测加权的作用是改变类的先验概率。
            
            {\color{blue} \bf{Missing Predictor Values}}
            
            {\color{blue} \bf{Why Binary Splits}}
            
            多路分裂会很快地把数据分裂成碎片,导致下一层的数据不足。而且多路分裂也可以由一系列二叉分裂组成。
            
            {\color{blue} \bf{Other Tree-Building Procedures}}
            
            CART(classification and regression tree)
            
            {\color{blue} \bf{Linear Combination Splits}}
            
            线性组合分裂可能增强树的预测能力,但可能破坏其可解释性。在计算方面,分裂点搜索的离散性阻碍了权值光滑优化的使用。
            
            {\color{blue} \bf{Instability of Trees}}
