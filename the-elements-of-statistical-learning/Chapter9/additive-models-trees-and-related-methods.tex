\chapter{Additive Models, Trees, and Related Methods}
    \section{Generalized Additive Models}
    \section{Tree-Based Methods}
        \subsection{Background}
        \subsection{Regression Trees}
        \subsection{Classification Trees}
            不同的节点非纯度度量${Q_m}(T)$,包括交叉熵或者deviance(散离)。

            交叉熵和基尼指数对结点概率的改变更加敏感,相对于错误率来说。
        \subsection{Other issues}
            {\color{blue} \bf{Categorical Predictors}}

            {\color{blue} \bf{The Loss Matrix}}

            观测的误分类后果对于某些类要比其他类严重。为了把损失引入到建模过程中,可以把Gini指数修改成$\sum\nolimits_{k \ne k'} {{L_{kk'}}} {\hat p_{mk}}{\hat p_{mk'}}$ 。该方法对多分类比较有效,对于二分类,系数不起作用,更好的办法是给$k$类中的样本加权$L_{kk'}$。对于多分类来说,仅当$L_{kk'}$与$k'$无关时才能使用。观测加权的作用是改变类的先验概率。

            {\color{blue} \bf{Missing Predictor Values}}

            {\color{blue} \bf{Why Binary Splits}}

            多路分裂会很快地把数据分裂成碎片,导致下一层的数据不足。而且多路分裂也可以由一系列二叉分裂组成。

            {\color{blue} \bf{Other Tree-Building Procedures}}

            CART(classification and regression tree)

            {\color{blue} \bf{Linear Combination Splits}}

            线性组合分裂可能增强树的预测能力,但可能破坏其可解释性。在计算方面,分裂点搜索的离散性阻碍了权值光滑优化的使用。

            {\color{blue} \bf{Instability of Trees}}

            树的方差较大,数据的一个较小变化将导致一系列完全不同的分裂,使得解释有些不稳定。这种不稳定性的主要原因是过程的分层本性,顶层分裂中的错误被传播到下面的所有分裂。

            {\color{blue} \bf{Lack of Smoothness}}

            可以看到预测面缺乏光滑性(分层)。在0/1损失的分类中,因为类概率估计中的偏倚的影响有限,因而不会产生太大伤害。然而,可能降低回归处理的性能。

            {\color{blue} \bf{Difficulty in Capturing Additive Structure}}
        \subsection{Spam Example(Continued)}
            交叉验证误差率由一系列α值来标引,而不是树的大小,因为对于同一个α,不同折生成的树可能大小不一样。(图9.4的交叉验证结果只被α值索引,测试结果既可被α,又可被剪枝后的树的大小索引)
            
            医学分类问题中,术语敏感性(sensitivity)(1预测为1)和特效性(specificity)(0预测为0)用来刻画规则。
            
            更好的方法不是仅在节点中修改贝叶斯规则(更改损失权重),而是在树增长过程中考虑不相等损失。
            
            ROC曲线下面的面积也被称为c-statistic,当考虑一个额外的预测子加在标准模型上的影响时,其可能不是一个合理的度量。新的预测子可能在模型散离度的改变上影响很大,而在c-statistic上只会小量的增大。另一方面,c-statistic在分析额外预测子对独立样本的分类的改变上有用。

